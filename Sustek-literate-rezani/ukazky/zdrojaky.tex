\input opmac
\typosize[8.5/10]
\topskip10pt

\setbox9\hbox{\tt p}
\output{\shipout\vbox to\vsize{\vss\box255\vskip5pt\vss}}

\tt
\pdffontexpand\font 0 200 2 autoexpand
\pdfadjustspacing2
\hyphenchar\font-1

% \hsize u Zpravodaje je 27cc, okraje rámu jsou 3mm, platí 27cc-6mm=329.61116pt
\let\oribegtt\begtt
\eoldef\begtt#1{\par\break
  \margins/1 (329.61116, \the\numexpr#1*10\relax) (0,0,0,0)pt
  \oribegtt}
\ttindent0pt

\def\tthook{\catcode`\^=14 \catcode`\/=0 }
\def\bylo(#1){{\localcolor\setcmykcolor{0 0 0 0.4}\setbox0\hbox{#1}\vrule height2.4pt depth-2pt width\wd0 \hskip-\wd0 \box0}}
\def\nove(#1){{\localcolor\Green#1}}
\def\nowe(#1){{\localcolor\Magenta#1}}
\def\Nove[#1]{\nove(#1)}
\def\Nowe[#1]{\nowe(#1)}


\begtt13
Procedure PridejDoMenu;
  Begin
  inc(Podnabidek[M]);
  if Poz=Nakonec then
    begin
    Nabidka[M,Podnabidek[M]]:=Text;
    if M<>Hlavni then Proc[M,Podnabidek[M]]:=Procedura;
    end
  else
    begin
    ...
    end;
  End;
\endtt

\begtt12
\def\nacislo#1 #2{\ifx-#1#20\else
    #2#1 \ifnum#2>99 \ifnum#2<200 \advance#2-100\fi\fi\fi}
\def\radeknabody#1;#2;#3;#4;#5;{%
  \def\no{#1}\def\bodyA{#2}\def\bodyB{#3}\def\bodyC{#4}\def\bodyD{#5}%
  \nacislo#2 \bodycA \nacislo#3 \bodycB
  \nacislo#4 \bodycC \nacislo#5 \bodycD}
\def\nvzpracujradek{%
  \ea\radeknabody\radek
  \sumac\bodycA \advance\sumac\bodycB
  \advance\sumac\bodycC \advance\sumac\bodycD
  \ifnum\bodycA=\bodyDQ
    \counta99 \sumac0
^  \else
^    \ifnum\bodycA=\bodyDNS
^      \counta99 \sumac0
^    \else
^      \counta90 \advance\counta-\sumac
^    \fi
^  \fi
^  \ifnum\sumac>0
^    \ea\advance\csname pocetnenul\kategorie\endcsname1
^  \fi
^  \getpartid
^  \Writeout{\the\counta:\surnameNA:\no:%
^    \bodyA:\bodyB:\bodyC:\bodyD:\the\sumac:}
^  \advance\countb1}
\endtt

\begtt16
% Makro |\radeknabody| zpracuje řádek a jednotlivé položky uloží
% do příslušných maker a čítačů.
/nove(%    \begin{macrocode})
\def\radeknabody#1;#2;#3;#4;#5;{%
  \def\no{#1}\def\bodyA{#2}\def\bodyB{#3}\def\bodyC{#4}\def\bodyD{#5}%
  \nacislo#2 \bodycA \nacislo#3 \bodycB
  \nacislo#4 \bodycC \nacislo#5 \bodycD}
/nove(%    \end{macrocode})
% Pokud je některý soutěžící diskvalifikován, do tabulky s~počtem bodů
% se mu zapíše z~prvního příkladu 666~bodů. Jeho součet bodů se nastaví
% na nulu a v~souboru \soub{results.srt?} bude uveden až na konci.
/nove(%    \begin{macrocode})
\def\nvzpracujradek{%
  \ea\radeknabody\radek
^  \sumac\bodycA \advance\sumac\bodycB
^  \advance\sumac\bodycC \advance\sumac\bodycD
^  \ifnum\bodycA=\bodyDQ
^    \counta99 \sumac0
^  \else
^    \counta90 \advance\counta-\sumac
^  \fi
^  \ifnum\sumac>0
^    \ea\advance\csname pocetnenul\kategorie\endcsname1
^  \fi
^  \getpartid
^  \Writeout{\the\counta:\surnameNA:\no:%
^    \bodyA:\bodyB:\bodyC:\bodyD:\the\sumac:}
^  \advance\countb1}
...
/nove(%    \end{macrocode})
\endtt

\begtt16
/nove(% \begin{macro}{\nvzpracujradek})
% Pokud je některý soutěžící diskvalifikován, do tabulky s~počtem bodů
% se mu zapíše z~prvního příkladu 666~bodů. Jeho součet bodů se nastaví
% na nulu a v~souboru \soub{results.srt?} bude uveden až na konci.
^% \changes{100307}{100307}{Ošetření diskvalifikací}
/nove(% \changes{100908}{100908}{Pročištění})
/nove(% \changes{140314}{140314}{Logika diskvalifikací})
/nove(% \changes{140314}{140314}{Přidáno top competitors})
^% \changes{140315}{140315}{Označení makra}
%    \begin{macrocode}
\def\nvzpracujradek{%
  \ea\radeknabody\radek
  \sumac\bodycA \advance\sumac\bodycB
  \advance\sumac\bodycC \advance\sumac\bodycD
  \ifnum\bodycA=\bodyDQ
^    \counta99 \sumac0
^  \else
^    \counta90 \advance\counta-\sumac
^  \fi
^  \ifnum\sumac>0
^    \ea\advance\csname pocetnenul\kategorie\endcsname1
^  \fi
^  \getpartid
^  \Writeout{\the\counta:\surnameNA:\no:%
^    \bodyA:\bodyB:\bodyC:\bodyD:\the\sumac:}
^  \advance\countb1}
...
%    \end{macrocode}
/nove(% \end{macro})
\endtt

{
\def\tthook{\catcode`\^=14 \catcode`\/=0 \def\){)} \advance\rightskip0.2pt}
\begtt24
^if eqtb[5847].int<>0 then eqworddefine(5847,0);
^if eqtb[5304].int<>1 then eqworddefine(5304,1);
^if eqtb[3412].hh.rh<>0 then eqdefine(3412,118,0);end;
{:1070}/nove({1075:}procedure boxend(boxcontext:integer/);var p:halfword;)
/nove(begin if boxcontext<1073741824 then){1076:}begin if curbox<>0 then begin
mem[curbox+4].int:=boxcontext;
if abs(curlist.modefield)=1 then begin appendtovlist(curbox);
if adjusttail<>0 then begin if 29995<>adjusttail then begin mem[curlist.
tailfield].hh.rh:=mem[29995].hh.rh;curlist.tailfield:=adjusttail;end;
adjusttail:=0;end;if curlist.modefield>0 then buildpage;
end else begin if abs(curlist.modefield)=102 then curlist.auxfield.hh.lh
:=1000 else begin p:=newnoad;mem[p+1].hh.rh:=2;mem[p+1].hh.lh:=curbox;
curbox:=p;end;mem[curlist.tailfield].hh.rh:=curbox;
curlist.tailfield:=curbox;end;end;
end{:1076}/nove(else if boxcontext<1073742336 then){1077:}if boxcontext<
1073742080 then eqdefine(-1073738146+boxcontext,119,curbox)else
geqdefine(-1073738402+boxcontext,119,curbox){:1077}/nove(else if curbox<>0)
/nove(then if boxcontext>1073742336 then){1078:}begin{404:}repeat getxtoken;
until(curcmd<>10)and(curcmd<>0){:404};
if((curcmd=26)and(abs(curlist.modefield)<>1))or((curcmd=27)and(abs(
curlist.modefield)=1))then begin appendglue;
mem[curlist.tailfield].hh.b1:=boxcontext-(1073742237);
mem[curlist.tailfield+1].hh.rh:=curbox;
end else begin begin if interaction=3 then;printnl(262);print(1065);end;
begin helpptr:=3;helpline[2]:=1066;helpline[1]:=1067;helpline[0]:=1068;
end;backerror;flushnodelist(curbox);end;end{:1078}/nove(else shipout(curbox/);)
/nove(end;{:1075}){1079:}procedure beginbox(boxcontext:integer);label 10,30;
^var p,q:halfword;m:quarterword;k:halfword;n:eightbits;
^begin case curchr of 0:begin scaneightbitint;
^curbox:=eqtb[3678+curval].hh.rh;eqtb[3678+curval].hh.rh:=0;end;
\endtt

\begtt14
^@ The global variable |cur_box| will point to a newly made box. If the box
^is void, we will have |cur_box=null|. Otherwise we will have
^|type(cur_box)=hlist_node| or |vlist_node| or |rule_node|; the |rule_node|
^case can occur only with leaders.
^
^@<Glob...@>=
^@!cur_box:pointer; {box to be placed into its context}
^
/nove(@) The |box_end| procedure does the right thing with |cur_box|, if
|box_context| represents the context as explained above.

/nove(@<Declare act...@>=)
procedure box_end(@!box_context:integer);
var p:pointer; {|ord_noad| for new box in math mode}
begin if box_context<box_flag then /nove(@<Append box |cur_box| to the current list,)
    /nove(shifted by |box_context|@>)
else if box_context<ship_out_flag then /nove(@<Store \(c/)|cur_box| in a box register@>)
else if cur_box<>null then
  if box_context>ship_out_flag then /nove(@<Append a new leader node that)
      /nove(uses |cur_box|@>)
  else ship_out(cur_box);
end;
\endtt
\break
}

{
\def\tthook{\catcode`\/=0 \def\/{/}}
\begtt20
\input opmac
\input gensrc
/nove(\SRCFILENAME) style.tex
\activettchar"
Format of the page is~A4 with~2cm~margins.
The basic font size is set to 12\,pt.
/nove(\BEGSRC)
\margins//1 a4 (2,2,2,2)cm
\typosize[12//14]
/nove(\ENDSRC)
Macro "\safedef" defines a control sequence which is not yet defined.
If it is already defined then its new definition is ignored.
/nove(\BEGSRC)
\def\safedef#1{\ifdefined#1\begingroup\afterassignment\endgroup\fi\def#1}
/nove(\ENDSRC)
Sections are defined in the same way as in~\OPmac.
/nove(\BEGSRC)
\safedef\section{\sec}
/nove(\ENDSRC)
\bye
\endtt
\break
}

\begtt22
\input gensrc
\def\SRChook{\longlocalcolor\Red}
\SRCFILENAME program.txt
Here we define a block which will be inserted to another place.
/nove(\BEGSRC<InternalLabel>{Displayed label})
second line
  /nowe(|<InnerBlock>)
/nove(\ENDSRC)
A block can be defined by parts.
/nove(\BEGSRC<InternalLabel>)
fourth line
/nove(\ENDSRC)
And here we insert the block.
\BEGSRC
first line
    /nove(|<InternalLabel>)
\ENDSRC
Finally we define the inner block.
/nowe(\BEGSRC<InnerBlock>{Inner block})
third line
/nowe(\ENDSRC)
\bye
\endtt

\begtt4
first line
    second line
      third line
    fourth line
\endtt

\begtt8
1190 let ra=15: gosub /nove(2080): input a 
1200 let ra=16: gosub /nove(2080): input b
1210 if b<a then let c=a: let a=b: let b=c
1220 let x=a: gosub /nove(3100): let d=f
1230 let x=b: gosub /nove(3100)
1240 if sgn(f)*sgn(d)<0 then /nove(1270)
1250 let ra=17: gosub /nove(2270)
1260 goto /nove(1390)
\endtt


\begtt20
\seccc Bisekce

Na začátku bisekce uživatel zadá krajní body do proměnných "A", "B".
\BEGSRC<bisekce>{Bisekce}
|vypisretezec(15): INPUT A |LL(bisekce)
|vypisretezec(16): INPUT B
\ENDSRC
Ze slušnosti je zařízeno, aby platilo "A"${}<{}$"B".
\BEGSRC<bisekce>
IF B<A THEN LET C=A: LET A=B: LET B=C
\ENDSRC
Pokud nejsou v~krajních bodech různá znaménka, vypíše se chybové hlášení a metoda se ukončí.
\BEGSRC<bisekce>
LET X=A: GOSUB |RR(deffce): LET D=F
LET X=B: GOSUB |RR(deffce)
IF SGN(F)*SGN(D)<0 THEN |RR(bis1)
|vypisretezecln(17)
GOTO |RR(bis0)
\ENDSRC
\endtt

\begtt6
\BEGSRC<SDR>{Concat solutions}
/nowe(SDRout)=/nove(SDRin2).clone();
if (/nove(SDRin1).max > /nove(SDRin2).max) {/nowe(SDRout).max = /nove(SDRin1).max;}
else {/nowe(SDRout).max = /nove(SDRin2).max;}
/nowe(SDRout).position = /nove(SDRin1).position.concat(/nove(SDRin2).position);}
\ENDSRC
\endtt

\begtt9
\BEGSRC
/nove(SDRin1)=mainsolution.clone();
/nove(SDRin1).position=mainsolution.position.clone();
/nove(SDRin2)=ZPRout.clone();
/nove(SDRin2).position=ZPRout.position.clone();
|<SDR>
mainsolution=/nowe(SDRout).clone();
mainsolution.position=/nowe(SDRout).position.clone();
\ENDSRC
\endtt

\begtt2
\SRConeline{second line}
\SRCblock{  }{InnerBlock}
\endtt

\begtt1
\SRConeline{fourth line}
\endtt

\begtt2
\SRConeline{first line}
\SRCblock{    }{InternalLabel}
\endtt

\begtt1
\SRConeline{third line}
\endtt

\begtt10
\DEFSRC {InternalLabel}{Displayed label}
\SRCbeginline :second line\SRCendline
\SRCreadblock {  }{InnerBlock}
\ENDDEFSRC 
\ADDSRC {InternalLabel}
\SRCbeginline :fourth line\SRCendline 
\ENDDEFSRC 
\DEFSRC {InnerBlock}{Inner block}
\SRCbeginline :third line\SRCendline 
\ENDDEFSRC 
\endtt

\begtt1
first line
\endtt

\begtt10
\expandafter\def\csname SRCtit:InternalLabel\endcsname{%
  Displayed label}
\expandafter\def\csname SRCcon:InternalLabel\endcsname{%
  \SRConeline{second line}%
  \SRCblock{  }{InnerBlock}%
  \SRConeline{fourth line}}
\expandafter\def\csname SRCtit:InnerBlock\endcsname{%
  Inner block}
\expandafter\def\csname SRCcon:InnerBlock\endcsname{%
  \SRConeline{third line}}
\endtt

\begtt1
\SRCcontent
\endtt

\begtt2
\SRConeline{first line}
\SRCblock{    }{InternalLabel}
\endtt

\begtt2
\SRCwriteline{first line}
\SRCwriteblock{    }{InternalLabel}
\endtt

\begtt6
\csname SRCwritehook\endcsname
\immediate\write\SRCfile{\SRCodsazeni first line}
\begingroup
\addto\SRCodsazeni{    }
\csname SRCcon:InternalLabel\endcsname
\endgroup
\endtt

\begtt8
\csname SRCwritehook\endcsname
\immediate\write\SRCfile{\SRCodsazeni first line}
\begingroup
\addto\SRCodsazeni{    }
\SRConeline{second line}
\SRCblock{  }{InnerBlock}
\SRConeline{fourth line}
\endgroup
\endtt

\begtt14
\csname SRCwritehook\endcsname
\immediate\write\SRCfile{\SRCodsazeni first line}
\begingroup
\addto\SRCodsazeni{    }
\csname SRCwritehook\endcsname
\immediate\write\SRCfile{\SRCodsazeni second line}
\begingroup
\addto\SRCodsazeni{  }
\csname SRCwritehook\endcsname
\immediate\write\SRCfile{\SRCodsazeni third line}
\endgroup
\csname SRCwritehook\endcsname
\immediate\write\SRCfile{\SRCodsazeni fourth line}
\endgroup
\endtt

\begtt24
\SRCFILENAME programA.txt
\BEGSRC
|<A>
\ENDSRC
\SRCFILENAME programB.txt
\BEGSRC
|<B>
\ENDSRC
\BEGSRC<A>{Program A}
|<firstpart>
in the first
|<secondpart>
\ENDSRC
\BEGSRC<B>{Program B}
|<firstpart>
in the second
|<secondpart>
\ENDSRC
\BEGSRC<firstpart>{First part}
This will be
\ENDSRC
\BEGSRC<secondpart>{Second part}
program.
\ENDSRC
\endtt

\begtt24
\SRCFILENAME programA.txt
\aBEGSRC
|<A>
\ENDSRC
\SRCFILENAME programB.txt
\bBEGSRC
|<B>
\ENDSRC
\aBEGSRC<A>{Program A}
|<firstpart>
in the first
|<secondpart>
\ENDSRC
\bBEGSRC<B>{Program B}
|<firstpart>
in the second
|<secondpart>
\ENDSRC
\BEGSRC<firstpart>{First part}
This will be
\ENDSRC
\BEGSRC<secondpart>{Second part}
program.
\ENDSRC
\endtt

{\def\tthook{\catcode`\:=0 }
\begtt7
\SRCFILENAME program\version.txt
\BEGSRC
select cast(NN/p as @:Nove[integer]@@), st+1
from FindN1,
    (select @:Nove[rownum]@:Nowe[p]@ r, p from SmallFactors) a
where st=r
\ENDSRC
\endtt
\break
}

{\def\tthook{}
\begtt4
select cast(NN/p as number(38)), st+1
from FindN1,
    (select rownum r, p from SmallFactors) a
where st=r
\endtt

\begtt4
select cast(NN/p as decimal(38)), st+1
from FindN1,
    (select (row_number() over (order by p)) r, p from SmallFactors) a
where st=r
\endtt
\break  
}

\begtt6
pdfcsplain soubor
pdfcsplain soubor
mv soubor.pdf souborOR.pdf
pdfcsplain '\def\version{MS}\input soubor'
pdfcsplain '\def\version{MS}\input soubor'
mv soubor.pdf souborMS.pdf
\endtt

\end
