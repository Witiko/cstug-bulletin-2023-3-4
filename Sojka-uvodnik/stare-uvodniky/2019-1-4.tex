Otevíráte první a poslední číslo Zpravodaje v roce 2019, čtyřčíslo s letošními články.  Po třech letech mi opět dovolte při příležitosti mého znovuzvolení předsedou našeho zapsaného spolku malé zastavení a zhodnocení stavu \CSTUG\unskip u a Zpravodaje.
\subsection*{Ohlédnutí zpět}
Jelikož jsem předsedal \CSTUG u kromě období 1995--2001 již poslední tři volební období (2010--2019), je na místě zhodnotit alespoň uplynulé tři roky.  Sdružení finišuje třetí dekádu své existence, a charakter sdružování se ve světě vysokorychlostního internetu a snadné digitální komunikace mění. Mnoho se od mého předání sdružení v roce 2001 kolegům Olšákovi (2001--2004) a Kubenovi (2004--2010) a za dobu mého předchozího předsedování změnilo.
Lokální skupiny uživatelů se spíše stávají profesními organizacemi, které pečují regionálně o dědictví a perspektivu oboru, v našem případě o dědictví a perspektivu digitální typografie sázecím systémem \TeX, resp. spíše pdf\TeX, který v našem regionu vznikl.
V úvodníku staronového předsedy ve {Zpravodaji 2016/1--4} jsem avizoval svůj program udržitelnosti sdružení a novou koncepci vydávání Zpravodaje.  Kromě dodržení všech zákonných povinností chodu neziskové organizace se podařilo i tuto koncepci a algoritmus vydávání čísel časopisu dodržet.  Dnes již dokonce máme převis nabízených článků a tak doufám, že další číslo Zpravodaje již najdete ve svých schránkách v první půli roku 2020.  Konsolidace vydávání našeho Zpravodaje stejně jako členské základny se tedy podařila.  Velké díky patří šéfredaktorovi Zpravodaje Honzovi Šustkovi a technickou redakci Zpravodaje skvěle zvládá Vít Novotný.  Věkový průměr autorů článků se pozvolna snižuje a vznikají smysluplné projekty hodné podpory, což mě naplňuje dalším optimismem ohledně udržitelnosti sdružení.  Podařilo se také přispět drobným dílem k návštěvě Grand Wizarda v Brně.
\subsection*{Valné shromáždění}
Letošní valná hromada proběhla již v říjnu, a byla volební.  Valné hromadě předcházela přednáška \emph{The Unreasonable Effectiveness of Patterns Generation}, kde jsem se synem zopakoval přednášku z konference TUG 2019, a prezentoval výsledky o zdokonaleném vývoji a generování nových českých vzorů dělení. Odpovídající článek najdete v tomto čísle Zpravodaje.
\subsection*{Poděkování}
Musím poděkovat předchozímu výboru a všem aktivním členům sdružení za to, že sdružení a jeho tradice byla udržena. Web Zpravodaje byl doplněn o čísla posledních let a články s referencemi byly registrovány u CrossRef, a jsou jim průběžně přidělována DOI pro snadnou a dlouhodobou odkazovatelnost.  I nadále byly udržovány diskusní listy a zpřístupňován archiv CTAN zrcadlením do Brna.
Setrvávající aktiva sdružení byla využívána na podporu \TeX ových projektů doma i ve světě. \CSTUG\ finančně podporoval projekty \TeX\ Gyre, tex4ht, a prezentaci projektu nových vzorů dělení na TUG 2019. Díky placení kolektivního členství \CSTUG u v TUGu bylo rozesíláno osm čísel časopisu TUGBoat do center \TeX ového života v Praze (knihovny MFF UK, FEL ČVUT a MÚ AV), Liberci, Ostravě (OU), Brně (Univerzita obrany), Bratislavě (MFF UK) a Košicích (UJŠ).
\subsection*{Zpravodaj}
Od svého založení \CSTUG\ nepřerušeně vydává od roku 1991 primárně pro komunikaci mezi svými členy Zpravodaj \CSTUG u.  Včasné vydávání se podařilo dodržet, při dodržení poměrně vysoké obsahové a formální kvality.  Nová pravidla vydávání avizovaná před třemi lety redakční rada dopracovala, a aktualizovala procesy redakce a pravidla recenzního řízení a otevřenost zapojení odborníků.  Jste-li ochotni přiložit ruku k dílu recenzemi, korekturami, či svým know-how, ozvěte se prosím redakci.
Podařilo se domluvit zařazení časopisu do Digitální matematické knihovny DML-CZ na \url{https://dml.cz}, kde by se měly všechny vydané články objevit během roku 2020.  Tím se dále zvýší viditelnost sdružení, dostupnost vydávaných článků a informací o digitální typografii sázecím systémem \TeX.
\subsection*{Editorial čísla}
\TeX ovou událostí roku je nepochybně návštěva profesora Knutha v Brně. To se podařilo zařazením provedení české premiéry Knuthova oratoria Fantasia Apocalyptica do oslav 25. výročí založení Fakulty informatiky Masarykovy univerzity, a pozváním prvního čestného doktora FI MU Donalda Ervina Knutha k opětovné návštěvě Brna.  O audiovisuálním představení referuje první článek tohoto čísla.
Stalo se již tradicí pořádat setkání uživatelů balíku \ConTeXt\ v ČR.  Na setkání vás fotoreportáží z předchozích setkání zve jménem \ConTeXt\ Group Jano Kula, a dalším článkem o sazbě tabulek v \ConTeXt u i Tomáš Hála.
Do hlubin vnitřností \TeX u se noří Lucie Schaynová a Jan Šustek.  Ve svém článku se zabývají vysvětlením způsobů, jakými lze modifikací output rutiny docílit specifických tvarů a efektů na stránce.
Jiří Rybička ve svém článku sdílí své dlouholeté zkušenosti s výukou zpracování textů na brněnské Mendelově univerzitě.
Problematice dělení slov a dechberoucí efektivity technologie vzorů dělení slov se zabývám v článku, který jsem napsal se svým synem.  Píšeme o přípravě nového, aktuálního, rozsáhlého a rozděleného seznamu slov pro generování kvalitních vzorů dělení slov.  Na příkladu češtiny jsme ověřili možnost technologie vzorů dělení připravit vzory, které nejen nechybují, ale dělí bezchybně celý seznam miliónů slov jazyka, přitom rychlost dělení je obrovská a paměťové nároky jsou minimalistické.
Dalším článkem čísla je zpráva Jana Kuly o verzích makrobalíku \ConTeXt.  Ukazuje směr, jakým se ubírá společné využití Lua\TeX u, \MP u a zmíněného makrobalíku.
Číslo se uzavírá překladem další části seriálu Petera Wilsona o tom, jak vkládat \MP ové obrázky do (pdf)\LaTeX ového dokumentu.
Prioritou mého předsednictví \CSTUG u zůstává zajištění alespoň minimálního chodu sdružení, pravidelné vydávání Zpravodaje a podpora aktivních jednotlivců a projektů naplňujících poslání sdružení.
Aktivity sdružení stojí a padají na ochotných jednotlivcích.  Máte-li co sdružení nabídnout, ať už v souvislosti s výše naznačenými činnostmi sdružení, nebo byste chtěli začít novou aktivitu v duchu stanov sdružení, napište prosím na adresu výboru z tiráže!
Budeme vděčni i za pomocnou ruku v získávání nových členů, propagaci sdružení, editaci a redakci tematických webových stránek na webu sdružení, ale i za konstruktivní náměty na další aktivity či připomínky ke stávajícímu chodu sdružení.
\subsection*{Dojmy a fotoreportáž z návštěvy u profesora Knutha v Palo Alto při TUG 2019}
Bylo mi potěšením a národním svátkem se zúčastnit konference TUG 2019 v rodišti \TeX u: v Palo Alto u campusu Stanfordské univerzity. Skoro dvě desítky účastníků pozval Don Knuth k sobě domů (univerzita svým profesorům půjčuje na campusu na 99 let dům). Do toho svého si Don pořídil vlastní varhany. Na dalších stránkách vidíte z návštěvy u Knutha a z konference krátkou fotoreportáž.
České sdružení uživatelů \TeX u jsem zastupoval se svým synem, kdy jsme prezentovali článek, který najdete v tomto čísle.  Zástupci \CSTUG ové komunity byli několikrát zmíněni v poděkováních, kromě prezentace Douga McKenny (viz obrázek \ref{fig:tug2019i}) to bylo poděkování Zdeňkovi Wagnerovi při prezentaci Shakthi Kannana.
