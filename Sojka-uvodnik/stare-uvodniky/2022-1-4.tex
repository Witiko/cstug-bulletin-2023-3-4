Je mi potěšením vám představit články tohoto čísla, a seznámit vás s informacemi ze světa našeho oblíbeného sázecího systému.
První článek čísla na straně \strankasclankem{Olsak-cropmarks/clanek-cropmarks} z pera kolegy Olšáka připomíná téma, před kterým stál každý, kdo někdy dělal předtiskovou přípravu tiskovin: téma ořezových značek a archové montáže. První makra pro ořezové značky na archívu CTAN pro plain \TeX\ jsou z let 1992 \cite{ungar1992cropmark} a 1994 \cite{taylor1994cropmrks}.  Že je problematika živá a potřebná ukazuje i makrobalík \makro{zwpagelayout} \cite{wagner2022zwpagelayout}.  Článek zmiňuje i další makrobalíčky vytvořené nejen pro \uv{intelektuální rozptýlení} autora, ale i jako \uv{námět pro hlubší rozpracování} \TeX ovými programátory.
Článek studentky Fakulty informatiky Masarykovy univerzity Terezy Vrabcové na straně \strankasclankem{Vrabcova-archivace/article} je záznamem její přednášky na jarním přednáškovém odpoledni \CSTUG\ \cite{cstug2021valna}: máme úplný digitální archiv celé historie našeho Zpravodaje od roku 1990, v digitální matematické knihovně \href{https:/dml.cz}{dml.cz}!  Včetně metadat obsahujících bibliografické záznamy, včetně přidělených persistentních identifikátorů DOI pro jednoznačné citování.  To umožňuje jednak dohledávat publikace téměř dvou stovek autorů publikujících ve Zpravodaji, ale také hledat souvislosti a podobné články v Evropské digitální matematické knihovně \href{https://eudml.org}{eudml.org}, kam je Zpravodaj dále exportován a archivován. Um celých generací českých a slovenských \TeX istů je tak zachován!
Mohou se kružnice líbat? V libovolné pozici, ve třech a rekurzivně?  To je téma článku Denise Roegela na straně \strankasclankem{Roegel-romantika/apo}, který se věnuje \MP u a implementaci nedávno objevených možností sestrojení Soddyho kružnice v něm.  Geometrovo oko nad deseti obrázky v článku zaplesá, a doufám, že zaplesá i vaše nad ukázanými možnostmi \MP u.
\TeX\ s výhodou používají programátoři jako Vítek Novotný, který se na jarním přednáškovém odpoledni \CSTUG\ \cite{cstug2021valna} a na straně \strankasclankem{Novotny-jazyky/main} v tomto čísle podělil o svůj pohled na možnosti vysokoúrovňového programování v \TeX ovém ekosystému.  Nic vám neříká Lua\-Meta\TeX, expl3, \textsc{YAML} nebo LyLua\-\TeX?  Chcete využít svých informatických schopností přepínání mezi různými úrovněmi abstrakce a oslovuje vás výzva Donalda Knutha v \TeX booku \emph{``Go forth now and create masterpieces of the publishing art!''}?  Pak je článek právě pro vás!
Kvalita algoritmu odstavcového zlomu, kdy \TeX\ řeší speciální variantu NP-úplného optimalizačního problému, byla vždy klíčovou výhodou sazby prováděné \TeX em.  Kromě mikrotypografických rozšíření, které typografické komunitě přinesl pionýrsky \THANH\ v pdf\TeX u, je podstatné \stress{automatizované} řešení nežádoucích parchantů, tedy vdov a sirotků.  To je téma článku Maxe Chernoffa na straně \strankasclankem{Chernoff-widows/zpravodaj} postaveného na možnostech programovacího jazyka Lua dostupného v Lua\TeX u.  Další potěšení pro programátory a zároveň typografy mezi námi!
Číslo rámuje překlad dvanáctého pokračování řady praktických doporučení Petera Wilsona \stress{Glisterings} na straně \strankasclankem{Wilson-glisterings/13} věnovaný rámečkování.
V den psaní tohoto úvodníku proběhlo již druhé přednáškové odpoledne organizované \CSTUG em. Odezněla, byla nahrána a na webu sdružení bude vystavena čtvrtá a pátá letošní přednáška \cite{cstug2022valna} objednaná sdružením, po třech jarních přednáškách \cite{cstug2021valna}.  V našich krajinách se děje mnoho důležitého vývoje pro československou i celosvětovou \TeX ovou komunitu, a \CSTUG\ se snaží tyto aktivity podporovat.
Daří se vydávat tento Zpravodaj, jehož všechny články od roku 1990 se podařilo s přispěním sdružení digitalizovat.  Po doplnění a kontrolách metadat včetně bibliografie a přidělení DOI u CrossRef jsou vystaveny a oindexovány v české digitální matematické knihovně \href{https://dml.cz}{dml.cz}.  Doufám, že s podporou sazby příspěvků šablonou na Overleaf, redakční rady a technické redakce je publikování v časopise již nyní potěšením a svátkem \smiley.  
Tato výše členských příspěvků umožní aspoň částečně pokrýt zvýšené výdaje dané dvěma dekádami navyšování nákladů spojených s chodem sdružení, které byly doposud kryty z finančních rezerv sdružení a dobrovolnickou činností několika nadšenců.  Umožní to rozdmýchat naději, že dosavadní poměrně minimalistický chod sdružení bude růst a rozšiřující se benefity vyplývající z členství povedou následně k růstu členské základny a komfortu uživatelů \TeX u.
Platba kolektivního členství \CSTUG u v \TUG\ nám přináší mnohá dobrodiní.
Závěrem chci poděkovat aktivním \TeX istům, autorům, členům sdružení uživatelů \TeX u.  Bez nich, převážně členů \CSTUG u a \TUG u a jejich \uv{labour of love} by to nebylo možné.
