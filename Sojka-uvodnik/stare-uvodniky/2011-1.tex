Kolega Kuben předsedal \CSTUG u již dvě volební období, a pro tuto nehonorovanou a zodpovědnou funkci hledal nástupce. Nikoho nového s mladickým elánem však nenašel. I když jsem ze svého předchozího vedení \CSTUG u v letech 1995--2001 věděl, o jak nevděčnou funkci se jedná, nakonec jsem s kandidaturou na předsedu sdružení svolil, a na prosincové valné hromadě byl zvolen novým předsedou sdružení.
Nelze než poděkovat předchozímu předsedovi a výboru za to, že sdružení a jeho tradice byla udržena, včetně základního programu činnosti. Nejen že byl udržován diskusní list cstex a zpřístupňován archiv CTAN zrcadlením do Brna, byl pravidelně vydáván Zpravodaj sdružení zejména péčí šefredaktora Zdeňka Wagnera, byla založena tradice setkání \TeX perience a péčí kolegy Stříže byly uspořádány první čtyři ročníky této akce.  Aktiva sdružení byla využívána na podporu \TeX ových projektů doma i ve světě -- fonty \TeX\ Gyre, \MP, Lua\TeX\ jsou na světě i díky finančnímu přispění \CSTUG u.
Před deseti lety byl \CSTUG\ v počtu členů lokálních sdružení uživatelů (LUG) na milion obyvatel na špici pomyslného pelotonu LUG. Dnes tomu tak již není, a zájem o členství má trvale sestupnou tendenci. Je tedy načase se ptát, zda \CSTUG\ nabízí to, po čem je mezi uživateli poptávka, či zda je snižování počtu členů dáno objektivními faktory jako je současná globální krize, zvýšenými možnostmi vyžití a uplatnění mimo občanská sdružení jako je \CSTUG\ apod.
Kolektivní členové většinou deponují ve svých knihovnách Zpravodaj sdružení, který dostávají v několika kusech, a takto přirozeně sdružují uživatele na jednotlivých pracovištích a městech. Většina kolektivních členů jsou školy či ústavy Akademie -- centra vzdělanosti, kde je \TeX\ používán k publikování nebo je tam i vyučován či jinak podporován.  Pokud víte ve svém okolí o organizaci, která \TeX\ používá, prosím navrhněte, aby zvážila své členství v \CSTUG u.
Líbí se mi tradice německy mluvících \TeX istů DANTE zvaná \uv{Stammtische}.  Jednou za dva týdny či za měsíc se ve některých městech pravidelně scházejí zájemci o \TeX\ a sdílejí své zkušenosti a znalosti.  I to je možná příčinou, že DANTE má více členů než celosvětový TUG.  Pokud byste ve svém městě chtěli tuto tradici založit, směle do toho a \CSTUG\ jistě pomůže dát místo setkání veřejnosti a svým členům ve známost.
Prioritou mého předsednictví na počátku třetí dekády činnosti sdružení bude stabilizace členské základny a zajištění aspoň minimálního chodu sdružení zajištěním zákonem daných povinností. Sem patří evidence členů, plateb členských příspěvků a odevzdání daňového přiznání. Základní činností sdružení pak je příprava a pravidelné vydávání Zpravodaje \CSTUG u. S členstvím \CSTUG u v CrossRef souvisí předávání metadat vydaných článků do této celosvětové databáze, a to zpětně pro články vydané od roku 1991, včetně seznamů citací. To by mělo zatraktivnit i publikování ve Zpravodaji pro akademiky.  Neméně důležitá bude prezentace \CSTUG u na Internetu, aktuální a udržované webové stránky, funkční administrativní rozhraní databáze členů, mailové diskusní skupiny a péče o ftp archiv CTAN a sdružení. Pokud ještě zbudou síly, tak pokračování tradice pořádání \TeX perience, podpora grantového systému \CSTUG\ a finanční podpora dalších projektů jako je Lua\TeX.
Aktivity sdružení stojí a padají na ochotných jednotlivcích.  Máte-li co sdružení nabídnout, ať už v souvislosti s výše naznačenými činnostmi sdružení, nebo byste chtěli začít novou aktivitu v duchu stanov sdružení, napište prosím na adresu výboru z tiráže!  I když to na první pohled nevypadá, agenda s vedením neziskové organizace jako je \CSTUG\ je nemalá -- můj archiv výborového listu od roku 1993 má 4083 zpráv a to zdaleka nearchivuji vše.
Budeme vděčni i za pomocnou ruku v získávání nových členů, propagaci sdružení, editaci a redakci tematických webových stránek na připravovaném novém webu sdružení, ale i za konstriktivní náměty na nové aktivity či připomínky ke stávajícímu chodu sdružení.
