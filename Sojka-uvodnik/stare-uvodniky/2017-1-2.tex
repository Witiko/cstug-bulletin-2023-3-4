Zafungovala druhá podmínka, v bufferu jsme měli článků skoro na trojčíslo, a některé již čekají na vydání v dalším čísle. Díky nové redakční radě za akvizici článků a jejich rychlé sesazení v době prázdnin.  Nebojte se posílat další články, sdílejte svá makra, své zkušenosti s užitím \TeX ových technologií na vašich školách, ve vašich firmách, na vašich pracovištích.
Všichni kolektivní členové a individuální členové, kteří si o ně napsali, mají k číslu přibaleno DVD \TeX{}live, opět s aktualizovanou česko-slovenskou dokumentací. Přestože mnozí členové se aktivně podílí a budou podílet na aktualizacích a přípravě hlavní \TeX ové distribuce, výbor sdružení se rozhodl již neposílat DVD paušálně všem členům, neboť většina má dostupné vysokorychlostní připojení a naopak nemá DVD mechaniku.
V čísle najdete několik příspěvků holandských kolegů Hanse Hagena a Taca Hoekwatera, vyvíjejících relativně nový makrobalík \Hologo{ConTeXt}, a experimentujících s možnostmi, které přidává imperativní jazyk Lua spolu s novými rozšířeními programu \MP. Čtenář tak může po přečtení své \MP\ obrázky ukládat ve formátu PNG, sázet Korán v arabštině po vzoru arabských kaligrafů, exportovat své dokumenty do formátů čteček elektronických knih nebo v jazyce Lua své dokumenty přímo programovat.
Praktické úloze, jak vložit Google mapu do dokumentu, se věnuje článek \p Mapy v \LaTeX ových dokumentoch -- predstavenie balíčka \texttt{getmap}/ Aleše Kozubíka.
Číslo končí zprávou z konference TUG@Bacho\TeX{} 2017.  Tu kromě autora příspěvku a autora úvodníku navštívili další tři členové našeho sdružení. Ze tří zde prezentovaných příspěvků členů \CSTUG u byla prezentace dvou podpořena finančně.  Dvoukonference byla dobře navštívena, jak vidíte na společné fotce účastníků v článku Víta Novotného.  Vše probíhalo v obvyklé srdečné atmosféře starých i nových přátel \TeX{}ové rodiny. Soudržnost a stálost rodiny ukazuje i to, že skoro desítka účastníků akce navštívila před čtvrt stoletím již konferenci Euro\TeX{} 92 v Praze, což byl velký impuls k rozšíření a rozkvětu TeXových aktivit v Československu.
Úvodník píši na malé plachetnici u Vancouveru, kde nás s technickým redaktorem tohoto čísla pan kapitán instruoval na co slouží písmeno T na velkých lodích a tankerech.  Je to místo, kam \textit{tug boat} -- malá, ale výkonná loď -- tlačí či tahá velkou loď s velkou setrvačností a tonáží tak, aby se držela či zakotvila na správném místě.
Přeji \TeX ovému tankeru, aby lokální skupiny uživatelů vytvářely malé, ale výkonné loďky, tugboaty či Zpravodaje, které by správným kotvením a směřováním čtyřicetiletého mohutného plavidla přinášely stále užitek.
