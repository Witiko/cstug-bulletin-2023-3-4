S radostí píši úvodník dalšího čísla Zpravodaje, který uzavře podle v minulém čísle avizovaného algoritmu vydávání ročníku 2017. Kromě potvrzení důležitosti snah o dodržení pravidelnosti vydávání Zpravodaje prosincová valná hromada schválila zrušení roční lhůty pro volné vystavení čísel Zpravodaje na webu sdružení.  Tím bude Zpravodaj dostupný rychleji větší čtenářské obci.  Doufáme tím v další zvýšení kreditu publikování ve Zpravodaji a \CSTUG u jako sdružení. Také apelujeme na stávající členy, aby ve svém okolí šířili povědomí o sdružení a pomáhali tak dobré věci, a naplnění cíle sdružení a tohoto Zpravodaje.
\CSTUG\ podporuje kvalitní typografii v ČR i SR přes čtvrt století, \TeX\ nedávno oslavil čtyřicátiny, a jeho autor osmdesátiny. Knuth je oslavil v kruhu svých přátel ve švédském Pite\r{a}, kde kromě odborného sympozia zazněla    premiéra Knuthova multimediálního díla \emph{Fantasia Apocalyptica}.  Detaily akce lze nalézt na \url{http://knuth80.elfbrink.se/}.  Knuth tím rozšířil řadu osob, kteří po větší část života používali \TeX.
V tomto čísle najdete tři články.  V prvním líčím své přesvědčení, že \TeX\ měl a má na školách typu Fakulta informatiky MU své místo, a nemalým způsobem přispívá ke kvalitě produkovaných dokumentů na školách.  Jen přínos vzniku pdf\TeX u na FI MU komunitě nelze jednoduše vyčíslit.
Že imperativní programování má smysl nejen v \TeX u, ale i v \MP u, přesvědčuje ve svém článku duchovní otec \textsc{Lua}\TeX u Hans Hagen. Možnosti programovatelné grafiky v \MP u tak dostávají nové možnosti a efektivitu.
Číslo uzavírá přeložený článek Petera Wilsona věnovaný možnostem formátování odstavců.  Zde si doufám najde použitelné triky každý uživatel \TeX u.
Všem čtenářům přeji pěkné chvíle nad dvojčíslem Zpravodaje.  Pro bezproblémovou sazbu v roce 2018 doporučuji opět volit \TeX.  Dle osobní zkušenosti je to dobrá, spolehlivá volba kvalitního sázecího systému, nejen na školách.  Který software každodenního užití vydržel čtyři dekády?  Přeji mu i jeho autorovi aspoň dvojnásobek.
