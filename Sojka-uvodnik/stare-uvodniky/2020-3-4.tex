I v dnešní svízelné době se lze radovat z drobností, jakou je toto druhé dvojčíslo Zpravodaje roku 2020.  Vzhledem k epidemickým opatřením jsme se nemohli v roce 2020 sejít na tradiční předvánoční přednášce a na valném shromáždění.  Výbor na vzniklou situaci reagoval návrhem změn stanov, které v budoucnu umožní korektně realizovat valné shromáždění i v online nebo hybridní podobě.  Podle starých stanov svolané valné shromáždění již navrženou změnu schválilo.  Volební valné shromáždění na podzim 2021 tak bude uspořádáno již podle nové verze stanov, kterou najdete na webu sdružení.  Uvítáme nominace a kandidaturu zejména nových agilních členů do výboru!
Valnému shromáždění předcházela přednáška Vítka Novotného o projektu digitalizace Zpravodaje.  V České digitální matematické knihovně (DML-CZ) již Vítkovým přispěním máme první ročník Zpravodaje a zbývajících 30 ročníků bude brzy následovat.  Online podoba přednášky zajistila vyšší účast než při kontaktní podobě v předchozích letech.  Své články v prvním ročníku Zpravodaje tak mohl vidět i zakládající předseda \CSTUG u:
Bohatství článků Zpravodaje tak bude dostupné dalším generacím \TeX ových nadšenců.  Snad se i zvýší motivace uživatelů v něm publikovat.
Na jednoho z prvních předsedů \CSTUG u a jeho využívání \TeX u a \MF u při sazbě časopisů v Matematickém ústavu AV ČR, pro sazbu učebnic a knih v dalších nakladatelstvích nebo pro organizační zajištění matematických olympiád vzpomíná v prvním článku čísla Pavel Stříž.  Karel byl \TeX ovým perfekcionistou a nadšencem \MF u, účastnícím se mnoha \TeX ových konferencí a projektů.
Jsme stále \emph{Československé} sdružení bez pomlčky, fungujeme společně, naše mateřské jazyky mají velmi podobnou morfologii i pravidla typografie, tak proč nemít i společné dělení slov?  Tuto minimalistickou snahu popisuji se synem v druhém článku tohoto čísla.  Nové společné vzory mají nejen větší přesnost a generalizační vlastnosti, ale jsou udržitelně generovány z vytvořené otevřené, aktuální a obrovské slovní zásoby obou jazyků zkompilované z veřejně publikovaných českých a slovenských textů na Internetu.
Minimalismus, použití nejnovějších nástrojů a efektivita bez kompromisů jsou nejen znaky designu závodních vozů formule 1, ale i principy, na kterých stojí návrh formátu Op\TeX{} Petra Olšáka.  V článku si přečtete výhody, které makrobalík skýtá těm uživatelům \TeX u, kteří ho znají pod kapotou do posledního šroubku, a mohou tak Op\TeX{} s výhodou využít pro své publikační projekty místo pohodlného sedanu \LaTeX u či Con\TeX tu.  Znáte-li \TeX\ jako Petr, přečtěte si jeho článek: můžete se těšit z minimalismu a rychlosti jeho nového závodního stroje.
Dobu mezi koronavirovými vlnami využili uživatelé makrobalíku Con\TeX t a sešli se v Sibřině u Prahy.  O akci referuje včetně fotoreportáže Jano Kula.
Dobré a špatné příklady makrodefinic jazyka \MP u prezentuje ve svém článku Taco Hoekwater.
Číslo tradičně uzavírá další přeložený díl seriálu funkčních ukázek sázecích technik od Petera Wilsona.
Doufám, že karanténa některým z vás umožní se soustředit na napsání článku o svých zkušenostech s \TeX em a příští číslo, již v DML-CZ, bude ještě obsáhlejší než toto a my budeme radostnější nejen z porážky viru, ale i ze sdílení svých typografických zkušeností a kvalitní typografie systémem \TeX.
