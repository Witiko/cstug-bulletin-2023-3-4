Otvíráte první a poslední číslo Zpravodaje v roce 2016, čtyřčíslo s články roku 2016. Dovolte mi při příležitosti mého znovuzvolení předsedou našeho zapsaného spolku malou reminiscenci a zhodnocení stavu \CSTUG\unskip u a Zpravodaje.
\subsection*{Ohlédnutí zpět}
Jelikož jsem předsedal \CSTUG u kromě období 1995--2001 již poslední dvě volební období (2010--2016), je na místě malé ohlédnutí zpět, a dovolím si pár myšlenek o našem sdružení.  Mnoho se od mého předání sdružení v roce 2001 kolegům Olšákovi (2001--2004) a Kubenovi (2004--2010) a za dobu mého předchozího předsedování změnilo.
Před šestnácti lety při mém předávání statutárské štafety byl \CSTUG\ v počtu členů lokálních sdružení uživatelů (LUG) na milión obyvatel na špici pomyslného pelotonu LUG. Dnes tomu tak již není, a zájem o členství má trvale sestupnou tendenci. Naše sdružení nevzkvétá.  Celosvětový TUG loni uspořádal roční kampaň s cenami za získávání nových členů. Členy se bez pobídek již nestávají běžní uživatelé \TeX u prostě proto, že nemají potřebu se organizovat ve spolku uživatelů jak bylo dříve běžné.  LUG se spíše stávají profesními organizacemi, kde zůstávají vesměs profesionálové \TeX em se živící, nebo dlouhodobí entuziasté středního či vyššího věku.
Je oprávněné se ptát, zda \CSTUG\ nabízí to, po čem je mezi desítkami tisíc řadových uživatelů \TeX u poptávka.  Je potřeba se ptát, jak plní své poslání a slouží členům.  Nasnadě je otázka, zda je snižování počtu členů dáno objektivními faktory, jako je snadná dostupnost informací a \TeX ových distribucí na Internetu, existencí podpory \TeX ových guru na platformách typu StackExchange, zvýšenými možnostmi vyžití a uplatnění mimo zapsané spolky a občanská sdružení, jako je \CSTUG, nebo neplněním očekávání členů od LUG.
V úvodníku staronového předsedy ve Zpravodaji 2011-1 jsem avizoval svůj minimalistický program sdružení: kromě dodržení všech zákonných povinností chodu neziskové organizace a dodržení slibů grantové finanční podpory jsem chtěl konsolidovat aspoň vydávání našeho Zpravodaje.  Nechávám na čtenáři posouzení, do jaké míry se to podařilo.  Nepodařilo se také řádně připomenout a oslavit čtvrtstoletí existence \CSTUG u.
Sebekriticky jsem pro volby na další období hledal pro nehonorovanou a zodpovědnou funkci statutára \CSTUG u svého nástupce. Přes veškeré úsilí se mi však nikoho nového s mladickým elánem k převzetí předsednické štafety přemluvit nepodařilo.  Naopak, u některých členů výboru to patrně vedlo k jejich rozhodnutí raději nekandidovat do výboru vůbec, a slíbili se dále angažovat nezávisle na závazku funkce v organizaci. Jelikož jsem ze svého předchozího dvanáctiletého vedení \CSTUG u věděl, o jak náročnou a zodpovědnou funkci se jedná, s kandidaturou do výboru -- a tedy s hrozbou návrhu na předsedu sdružení -- jsem dlouho váhal.
\subsection*{Valné shromáždění}
Na obvyklé přednášky před každoročním valným shromážděním se mi podařilo sehnat hned dvě přednášky svých studentů na Fakultě informatiky MU: Bc.\ Víta Novotného a Bc. Dávida Luptáka. Po uvedení a vyslechnutí obou přednášek (články k oběma jsou v tomto čísle) a po rozhodnutí obou přednášejících se stát aktivními členy sdružení jsem pojal důvěru v budoucnost spolku založenou na šikovných a iniciativních mladých členech.  A poté, co Vít Novotný na shromáždění vyjádřil i ochotu pomoci s dosud váznoucí technickou redakcí Zpravodaje \CSTUG\ a aktivně se nabídl přiložit ruku k dílu Honza Šustek, bylo rozhodnuto. Na následné schůzi výboru jsem s podmínkou pozitivního obratu a změn koncepce ve vydávání Zpravodaje souhlasil s opětovným zvolením do čela spolku a stal se tak staronovým předsedou sdružení.
\subsection*{Poděkování}
Nelze než poděkovat předchozímu výboru za to, že sdružení a jeho tradice byla udržena, včetně minimalistického programu činnosti. Nejen že byly udržovány diskusní listy a zpřístupňován archiv CTAN zrcadlením do Brna, byl vytvořen nový web sdružení. Částečně byl redukován skluz ve vydávání Zpravodaje sdružení, zejména péčí jeho šéfredaktora Zdeňka Wagnera a jeho zástupce Tomáše Hály.
Setrvávající aktiva sdružení byla využívána na podporu \TeX ových projektů doma i ve světě. \CSTUG\ finančně podporoval projekty fontů \TeX\ Gyre, \MP, Lua\TeX. Ty jsou na světě i díky finančnímu přispění \CSTUG u. Díky placení kolektivního členství \CSTUG u v TUGu bylo rozesíláno osm čísel časopisu TUGBoat do center \TeX ového života v Praze (knihovny MFF UK, FEL ČVUT a MÚ AV), Liberci, Ostravě (OU), Brně (Univerzita obrany), Bratislavě (MFF UK) a Košicích (UJŠ).
\subsection*{Zpravodaj}
Od svého založení \CSTUG\ vydává primárně pro komunikaci mezi svými členy Zpravodaj \CSTUG u.  Nedávno v tichu oslavilo periodikum čtvrtstoletí svého nepřerušeného vydávání.  V předchozím období se nedařilo redakci časopis vydávat pravidelně, jednak z důvodu nedostatku nabízených kvalitních příspěvků, jednak kvůli technické náročnosti sesazení jednotlivých čísel.  Své sehrálo i pracovní vytížení protagonistů redakce, šéfredaktora a jeho zástupce, vzhledem k vysokým ambicím, které se ale včas nedařilo naplňovat a komunikovat.
\subsection*{Nová koncepce vydávání Zpravodaje}
Vzhledem ke své pracovní zaneprázdněnosti na pozici šéfredaktora Zpravodaje rezignoval Zdeněk Wagner.  Hozenou šéfredaktorskou rukavici zvedl Honza Šustek, který byl v této roli jednomyslně výborem potvrzen.  Důležitý bod jeho koncepce vydávání jde vstříc autorům, kteří na redakční zpracování svých článků `v bufferu' redakce dříve čekali nezvykle dlouho.
Plně se s tímto programem ztotožňuji a toto číslo připravené v rekordně krátkém čase je prvním hmatatelným důkazem jeho naplňování. Díky!  Věřím, že příslib pravidelnosti přivede k zasílání článků nové autory, kteří se podělí o své zkušenosti či zpracují problematiku formou přehledového článku.
Na Valném shromáždění, jakožto nejvyšším orgánu spolku, byla tato problematika diskutována.  V diskusi s přijetím role technického redaktora Zpravodaje k mé velké radosti souhlasil Vít Novotný.  Že to byl dobrý krok svědčí i číslo, které držíte v rukou.
Příkaz \texttt{svn log|grep vnovotny|wc -l} informující o počtu revizí, které na tomto čísle od svého vstupu do \CSTUG u před valným shromážděním udělal, dává tříciferné číslo.  To demonstruje objem práce, který sesazení a redakce takového čísla obnáší. Případné nejednotnosti jdou na můj vrub daný tlakem číslo rychle dostat ke členům a srovnat skluz ve vydávání.
Nová pravidla vydávání sice bude ještě třeba kolektivně dopracovat, zajistit transparentnost procesů redakce, pravidla recenzního řízení a otevřenost zapojení odborníků při technických výzvách sazby a přípravy Zpravodaje do tisku. Jste-li ochotni přiložit ruku k dílu recenzemi, korekturami, či svým know-how, ozvěte se prosím redakci.
\subsection*{Editorial čísla}
Obsah tohoto čísla začíná dvěma články v angličtině referujícími o projektech, které \CSTUG\ podpořil finančně: \TeX\ Gyre a SWIGLIB.
Polští kolegové podávají zevrubnou zprávu o tom, jak po dvě dekády přispívali do pokladnice volně šiřitelných fontů, včetně vývoje infrastruktury umožňující i vývoj matematických fontů a OpenType verzí fontů. Dočtete se o historii Latin Modern fontů, důležitosti kompatibility metrik fontů, vývoji rodiny fontů \TeX\ Gyre nahrazující komerční rozšířené alternativy základních fontů Adobe, a doplňující pět rodin fontů kompletní podporou \TeX ové matematiky. Závěrem jsou naznačeny směry dalšího vývoje.
Na výborové schůzi \CSTUG u byl odsouhlasen závazek další vývoj fontů GUST E-foundry v letech 2017--2019 finančně podporovat částkou 1000 Eur ročně.
Luigi Scarso referuje o svém projektu SWIGLIB, taktéž podpořeném finančně \CSTUG em. Projekt umožňuje efektivní využití binárních knihoven v Lua\TeX u, zachovávaje standardní devizy \TeX ových řešení, a to efektivitu a portabilitu.
Marek Pomp sdílí své zkušenosti s literárním programováním v jazyce R vhodném pro statistické výpočty a přípravu diagramů.
Vít Novotný popisuje svůj nový makrobalík Markdown umožňující stejnojmenný jazyk použít v kterýchkoli dokumentech \LaTeX u, plain \TeX u a Con\TeX tu. Poměrně rozšířený a úsporný jazyk tak nyní lze sázet kvalitním sázecím strojem, a také ve složitějších dokumentech kombinovat se složitějšími objekty nepodporovanými ve značkování Markdown. Článek demonstruje vhodnost využití dostupnosti Lua interpretu místo makroexpanze pro implementaci řešení  problému parsování textu v jazyce Markdown.
Problematikou elektronických knih a jejich přípravy z klasicky značkovaných dokumentů se zabývá článek Michala Hofticha.  Řešení je postaveno na konverzi do XML formátů elektronických čteček pomocí programu \TeX4ht.
Neméně atraktivní a v komunitě žádanou funkcionalitou je sazba bibliografických citací vyhovujících normě ISO 690.  V článku o balíku \textsf{biblatex-iso690} \biblatex u Dávid Lupták informuje o výsledcích své bakalářské práce na Fakultě informatiky MU, která umožňuje splnit požadavky této nedávno aktualizované normy.  O poptávce a významu práce svědčí rychlost, s jakou  developerská komunita po jeho řešení sáhla, měřená počtem větví v repozitáři projektu. 
Číslo se uzavírá překladem další části seriálu Petera Wilsona o využití \LaTeX o\-vých nástrojů pro cykly na práci s řetezci.
\subsection*{Poděkování členům, zejména kolektivním}
Prioritou mého předsednictví v druhém čtvrtstoletí \CSTUG u bude kromě zajištění minimálního chodu sdružení stabilizace vydávání Zpravodaje a podpora aktivních jednotlivců a projektů naplňujících poslání sdružení.
S členstvím \CSTUG u v CrossRef souvisí předávání metadat vydaných článků do této celosvětové databáze, a to zpětně pro články vydané od roku 1991, včetně seznamů citací. To by mohlo zvýšit viditelnost Zpravodaje a zatraktivnit i publikování ve Zpravodaji pro akademiky.  Neméně důležitá bude prezentace \CSTUG u na Internetu, aktuální a udržované webové stránky, funkční administrativní rozhraní databáze členů, mailové diskusní skupiny a péče o ftp archiv CTAN a sdružení. 
Aktivity sdružení stojí a padají na ochotných jednotlivcích.  Máte-li co sdružení nabídnout, ať už v souvislosti s výše naznačenými činnostmi sdružení, nebo byste chtěli začít novou aktivitu v duchu stanov sdružení, napište prosím na adresu výboru z tiráže!
Budeme vděčni i za pomocnou ruku v získávání nových členů, propagaci sdružení, editaci a redakci tematických webových stránek na webu sdružení, ale i za konstruktivní náměty na další aktivity či připomínky ke stávajícímu chodu sdružení.
Líbí se mi tradice německy mluvících \TeX istů DANTE zvaná \uv{Stammtische}.  Přibližně jednou za měsíc se ve některých městech u jednoho stolu pravidelně scházejí zájemci o \TeX\ a sdílejí své zkušenosti a znalosti.  I to je možná příčinou, že DANTE má více členů než celosvětový TUG. Pokud byste ve svém městě chtěli tuto tradici osobních kontaktů založit, směle do toho a \CSTUG\ jistě pomůže dát místo setkání svým členům ve známost. Jen do toho! 
