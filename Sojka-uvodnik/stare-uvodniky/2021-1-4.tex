Je mi potěšením, že se redakci Zpravodaje daří získávat odborné články vyjadřující poslání sdružení, tedy kvalitní typografii systémem \TeX, že se stále nástroje a technologie posunují a můžete se o nich na stránkách tohoto čísla dozvědět, že i naše malá komunita k nim přispívá nemalým dílem, a že k nim také mohu přidat pár slov.
Články tohoto čísla ukazují, že sázecí systém \TeX\ se má stále čile k světu, spěje k dokonalosti, jeho technologie jsou široce používány, stále adaptovány a zdokonalovány, a upevňuje si pozici pevného bodu ve světě vědeckého publikování.
Kolaborativní práce na dokumentech výrazně urychluje učicí křivku počítačové typografie sázecím systémem \TeX.  Systém Overleaf toto umožňuje: ve sdíleném úložišti v reálném čase sázet dokumenty, sdílet plnou instalaci \TeX u včetně historických verzí distribuce \TeX live, nebo jen sdílet dokument s možností ho připomínkovat.  Již šest miliónů uživatelů systém používá, a v prvním článku čísla o něm a jeho použití na Fakultě informatiky Masarykovy univerzity (FI MU) píše Vítek Novotný.  Jako dlouholetý uživatel systému ho mohu pro kolaborativní editaci společných článků v \LaTeX u jen doporučit.  Doufám, že jeho použití najdou i čtenáři tohoto čísla.
Číslo pokračuje článkem shrnujícím pohledem Petra Olšáka vše potřebné pro \TeX ového, \eTeX ového, \XeTeX ového či Lua\TeX ového programátora.  Potěší všechny, kteří chtějí mít na úrovni \TeX u vše pod kontrolou, a porozumět do detailů všem primitivům, které jsou utajeny pod slupkou logického značkování \LaTeX u či Con\TeX tu.  Jako vedlejší produkt článku vznikl styl sazby článku do Zpravodaje v OP\TeX u, tak snad se brzy dočkáme dalších článků v tomto formátu.
Autor \TeX u po sedmi letech prošel 250, Karlem Berrym již předtřízených, chybových reportů k \TeX u a \MF u!  Grand Wizard ve svém článku připomíná svou neměnnou filosofii vývoje obou programů, komentuje je a informuje o vystavení šeků odměn za nalezení chyb. Jde o částky v součtu v řádu tisíců dolarů -- tak velká touha po dokonalosti se tu zrcadlí, a tak složité je psaní rozsáhlého software!  Máme novou verzi obou programů, o jedno desetinné místo bližší k $\pi$ resp. $e$.  Snad je 957. chyba již poslední, odhalení dalších lze případně čekat v roce 2029!
Se svými celoživotními zkušenostmi z praxe člena technické podpory Americké matematické společnosti se dělí Barbara Beeton v článku \uv {Ladění \LaTeX ových souborů}.  Navržené postupy a nástroje pro diagnostiku chyb uvítá každý autor netriviálních dokumentů.  Není potřeba znovunalézat kolo!
Jazyk Markdown si získal mezi značkovacími jazyky široké užití zejména mezi programátory pro svou kompaktnost.  Balíček Markdown již pět let umožňuje milovníkům Markdownu a \TeX ovým mágům přímou sazbu v \LaTeX ovém prostředí.  Nové rozšíření balíčku umožňuje definovat témata, styly chování, a jejich nastavení (snippety) aplikovatelné na části textu. Jen houšť!
S Vítkovým článkem souvisí i další článek dalšího studenta FI MU, Dominika Reháka.  Navrhovaná řešení umožní s typografickou kvalitou \TeX u s využitím balíčku Markdown sázet dokumenty všech dokumentových formátů podporovaných konvertorem \emph{Pandoc}.  To může podpořit kvalitní typografii systémem \TeX\ u široké palety autorů píšících v HTML, v jednoduchých značkovacích jazycích v jednoduchých textových editorech, či v méně zdatných sázecích systémech podporovaných programem \emph{Pandoc}.
Miniknihy, povolené taháky, leporela, skládačky vám pomůže vytvořit přeložený článek Petera Wilsona.  Toto jedenácté pokračování Wilsonova seriálu Zpravodajové čtyřčíslo uzavírá.  Nechť vám je tvorba vlastních miniaturních knížek a četba celého čísla potěšením.
Letošní vlny kovidu stále nepřejí osobnímu setkání na každoroční valné hromadě.  Díky pozměněným stanovám se tedy po necelém roce po Novém roce sejdeme online.  Detaily budou včas rozeslány elektronicky.  Valná hromada bude volební, budete si volit nový výbor.  Hledáme další nové, neunavené, nadšené kandidáty do výboru, kteří by naši malou československou \TeX ovou loďku tlačili další tři roky.  Podobně, podělte se s ostatními o své projekty a nápady nabídkou přednášky při valné hromadě nebo článku do Zpravodaje.  Pište nám, prosím, na adresu listu výboru nebo mně osobně.  Brzy na viděnou!
