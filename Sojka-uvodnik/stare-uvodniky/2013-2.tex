Jak patrno již z obálky, leitmotivem tohoto čísla je článek o rodině písem \emph{kp}, o \emph{kpfonts}.  Příběh o historii této rodiny fontů je ukázkou možností a schopností jak lidského ducha, tak otevřených technologií a software.
Příběh autora článku je příběhem běžného středoškolského učitele matematiky, který se vlastním dlouholetým úsilím, pílí a postupným sebevzděláváním dopracoval na autora jedné z největších, nejflexibilnějších rodin písem.  Návrh písem, zvláště pak i matematické sady znaků, je přitom považován za svatý grál grafického designu, a autor na fontech pracoval a zákonitosti jejich designu se učil ve svých volných chvílích ve svém středním věku.  To vše by nebylo možné bez otevřenosti a dostupnosti všech softwarových nástrojů a balíků volně poskytnutých autory jako \uv{Labour of Love}, a bez jejich dostupnosti v archivech jako je CTAN a distribucích jako je \TeX live.  To dává možnost tvořivé autory žádat o více a více, a příběh o \emph{kp} fontech ukazuje, že i na první pohled nerealistické požadavky mohou být vyslyšeny a uskutečněny.
Článek zaujme i ukázkami variant matematické sazby a bohatostí variant řezů fontů a jejich podporou v NFSS \LaTeX u.
Článek kolegy Šustka ilustruje možnosti \MF u pro vytváření barevných obrázků a práci s vrstvami.  I když podobné věci byly v roce 1977, kdy Knuth \MF\ navrhoval, mimo představivost většiny uživatelů, dnes se nad podobnými možnostmi již nikdo příliš nepozastavuje.
Překlad článku Petera Wilsona je dalším příspěvkem pro začínající uživatele \LaTeX u prezentující tři zajímavé problémy a jejich řešení.
Tomáš Hála píše o atraktivní a žádané problematice sazby bibliografických citací podle platné a poměrně nedávno aktualizované normy ČSN ISO 690.  Vím, že na její podpoře formou otevřeného software se již pracuje a doufám v její zveřejnění v některém z příštích čísel Zpravodaje.
Závěrem zmiňuji v návaznosti na články v čísle citát Pieta Heina.  Ten Donald Knuth považoval za tak důležitý a charakteristický pro svůj přístup k sazbě, programování a životu, že si ho nechal vytesat do kamene tvaru superelipsy u vstupních dveří jeho domu, aby ho měl stále na očích. Čtenářům přeji rychlé iterování podle tohoto citátu, a to nejen při sazbě \TeX em.
