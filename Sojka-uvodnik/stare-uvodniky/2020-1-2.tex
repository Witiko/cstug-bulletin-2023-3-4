Jsem rád, že dostáváte a otevíráte včas první letošní dvoučíslo Zpravodaje, plné zajímavých informací ze světa kvalitní typografie sázecím systémem \TeX.  Dovolte mi několik poznámek k obsahu čísla, a zejména pak k loňské návštěvě Grand Wizarda, tedy autora \TeX u Donalda Ervina Knutha, v Brně.
Toto číslo začíná článkem o novém balíku na generování zadání písemek.  Přesto, že podobná úloha již byla v několika balících řešena, článek ilustrativně a názorně ukazuje výhody, které takové cvičení pro autora a řešení jeho potřeb přináší, kromě seznámení se s makroprogramováním.  Taková řešení umožňují řešení přesně na míru a dobrou udržitelnost v čase. Po podobném řešení dříve nebo později sáhne většina pedagogů, a poptávka po nich v dnešní koronavirové době a potřebě distančního zkoušení roste. Na \TeX u jsou postavena pro jeho algoritmizovatelnost mnohá řešení: informační systém MU například generování písemek podporuje již víc než dekádu a je rutinně využíván tisíci zaměstnanci Masarykovy univerzity.
Další článek Vítka Novotného posunuje zájemce o minimalistické značkování na trůn, za který je autorem a širokou komunitou uživatelů považován jazyk Markdown. V článku se dozvíme jak \TeX\ a Lua spolu spolupracují na sazbě Markdown dokumentů ať již z příkazové řádky nebo při sazbě \TeX ových dokumentů s částmi v Markdown.
Jak načíst a zpracovat dat vytvořená v tabulkovém editoru ukazuje Honza Šustek, jako příklad makroprogramování na úrovni Plain\TeX u.
V říjnu 2019 na pozvání Fakulty informatiky Masarykovy univerzity (FI MU) navštívil Brno sám autor \TeX u, Grand Wizard, Donald Ervin Knuth (DEK).  Měl jsem tu čest ho po dobu jeho desetidenního pobytu v Brně provázet a koordinovat přípravy provedení české premiéry jeho varhanního opusu Fantasia Apocalyptica, kvůli které primárně přijel.
Záznam české premiéry varhanního oratoria Fantasia Apocalyptica si můžete poslechnout na adrese \url{https://youtu.be/wk7dEKMPP68}.
DEK měl na FI MU dvě přednášky, které vám formou transkriptu přináší článek Tomáše Szaniszla. Ač jsem Knutha provázel již v roce 1996 při udílení čestného doktorátu od MU a loni ho navštívil při TUG 2019 u jeho varhan doma v Kalifornii, z každé z obou brněnských přednášek jsem si odnesl několik nových, jen tak bokem zmíněných, mouder, které si připomínám a reflektuji ještě mnoho let poté.
Pozitivní afinita při aktivním hledání svých chyb a radost při jejich nalezení, začínání dne těmi nepříjemnými, ale nutnými úkoly, hledání maximální diverzity při výběru svých činností, získání maximální koncentrace a výkonu minimalismem přepínání činností a prací v dávkách, minimalizací komunikace (emailů), a při tom všem obrovské soustředění a laskavost na okolí, se kterým komunikuje kontaktně.
Místo několikáté návštěvy varhan DEK raději volil návštěvu Muzea romské kultury.
Doufám, že si svou inspiraci hodnou následování v transkriptech najdete také!  Obrázek o atmosféře přednášek si můžete udělat z přiložené fotoreportáže; fotky z 8.\,a 9.\,10. jsou dílem Martiny Morávkové, ostatní jsem fotil sám.
Číslo uzavírají přeložené ukázky ze seriálu \emph{Glisterings} Petera Wilsona, tentokráte věnované variantám způsobů opakování textu číslovaných prostředí.
