Je mi potěšením uvést malinké tematické číslo Zpravodaje věnované návrhu a produkci knih, klasických či elektronických. Došlo totiž k pozitivnímu obratu v získávání a dohledávání vhodných článků pro náš časopis.  Přes problémy zmíněné v úvodníku prvního letošního dvojčísla obálka tohoto čísla ukazuje, že vydávání Zpravodaje ještě hrana neodzvonila.  Úsilím odvážné redakce se podařilo pro potěšení čtenářů, ale i pro uspokojení knihoven dostávajících povinné výtisky periodických publikací a kontrolujících periodicitu vydávání připravit dvojčíslo, které držíte v rukou.
Článek Martina Peciny potěší každého se vztahem ke klasickým knihám, zejména ty, kdo si do postele na čtení raději berou tištěnou knihu místo moderních chladných nebo hřejivých elektronických přístrojů.
Ani druhý článek nevyžaduje žádné technické znalosti \TeX u.  Čtenář se dozví kdo, jak a proč přiděluje knihám čísla známá jako ISBN, a jak je tento mechanismus rozšířen a provozován v ČR i pro knížky elektronické.
Naopak hodně technický je článek Luigiho Scarsa, jehož práci pro projekt SWIGLIB drobně finančně podpořil \CSTUG. Reaguje na potřeby optimalizace a zrychlení sazby velkých knih a publikačních projektů, typicky obrázkových katalogů či rozsáhlých výukových databází.  Ty jsou typicky plné grafických dat dodávaných a aktualizovaných v různých formátech a je třeba je při databázovém publikování s využitím \TeX u  (přesněji v makrobalíku \ConTeXt{} nad Lua\TeX em) opakovaně a rychle zpracovávat.  Demonstruje možnosti spojení světů a až dábelsky složité kombinace makroprogramování v \TeX u a imperativního programování v jazyce Lua generujícího kód v zásobníkovém jazyce PostScriptu. Ukazuje to, jak inherentně složitá problematika sazby knih je, pokud ji chceme maximálně automatizovat.
Závěrem se přidávám k apelu z úvodníku minulého čísla: posílejte prosím redakci své články, \emph{zprávy} a \emph{daj}te je tak ostatním \TeX ovým nadšencům i k dispozici, sdílejte je s nimi.  Jak ukazují první dva články tohoto čísla, nemusí jít o technickou makroekvilibristiku, naopak žádané jsou články o praktickém sdílení široce pojímaných zkušeností a znalostí nástrojů a postupů použitelných pro kvalitní typografii sázecím systémem \TeX, tedy o to, o co nám v \CSTUG u jde primárně.
