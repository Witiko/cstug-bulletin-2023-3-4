\let\oldlooseness=\looseness
\RequirePackage{luatex85}
\PassOptionsToPackage{shorthands=off}{babel}
\makeatletter
\disable@package@load{fontenc}
\makeatother
\documentclass{csbulletin}
\selectlanguage{czech}
\usepackage[utf8]{inputenc}
\usepackage{luavlna}
\usepackage[strict]{lua-widow-control}
\usepackage{csquotes}
\usepackage[
  backend=biber,
  style=iso-numeric,
  sortlocale=cs,
  autolang=other,
  bibencoding=UTF8,
  mincitenames=2,
  maxcitenames=2,
]{biblatex}
\addbibresource{uvodnik.bib}
\usepackage[
  implicit=false,
  hidelinks,
]{hyperref}

% Pomocná makra
\def\TUG{TUG}
\def\TUGboat{TUGboat}
\let\stress\emph
\let\makro\texttt
\makeatletter
\DeclareRobustCommand{\La}{L\kern-.36em%
        {\sbox\z@ T%
         \vbox to\ht\z@{\hbox{\check@mathfonts
                              \fontsize\sf@size\z@
                              \math@fontsfalse\selectfont
                              A}%
                        \vss}%
        }}
\makeatother
\def\AllTeX{(\La\kern-.075em)\kern-.075em\TeX}

\begin{document}

% Metadata
\title{Úvodník}
\EnglishTitle{Editorial}
\author{Vít Starý Novotný}
\podpis{Vít Starý Novotný, witiko@mail.muni.cz}
\maketitle

Milé čtenářky a~čtenáři, \TeX istky a \TeX isté!

\medskip

Vítejte u druhého letošního dvoučísla Zpravodaje \CSTUG u, ve kterém pokračujeme ve snaze poskytovat cenné informace a poznatky ze světa \TeX u. 45 let od svého vzniku si \TeX{} udržuje své nezastupitelné místo jako nástroj pro kvalitní sazbu dokumentů, a my jsme hrdí na to, že můžeme být součástí jeho rozvoje.

Číslo Zpravodaje, které držíte v rukou, obsahuje řadu zajímavých článků:

\begin{itemize}
  \item \textbf{\CSTUG{} na konferenci TUG 2023} od autora tohoto úvodníku popisuje letní účast členů \CSTUG u na konferenci TUG 2023 v Bonnu.
  \item \textbf{Generování dokumentovaného zdrojového souboru po blocích v \TeX u} od Honzy Šustka přináší nový pohled na tvorbu dokumentovaných zdrojů. Článek zpracovává téma, o kterém Honza přednášel na konferenci OSS Conf 2019 v Žilině a na TUGu 2023. Honza bude o tématu článku přednášet také na valném shromáždění \CSTUG u, které se koná v sobotu 16.~prosince 2023 v Brně~\cite{starynovotny2023valna}.
  \item \textbf{Jak umožnit stránkový zlom uvnitř vložených obrázků}, také od Honzy Šustka, který se zabývá řešením specifických sazebních výzev.
  \item \textbf{Markdown 3: Co je nového a co nás čeká?} od autora tohoto úvodníku pojednává o nejnovějších trendech v jazyce Markdown. Článek zpracovává téma, o kterém Vítek přednášel na TUGu 2023.
  \item \textbf{Plán pro univerzální slabičnou segmentaci} Ondřeje Sojky, Petra~Sojky a Jakuba Mácy ukazuje možnosti univerzálního dělení slov nejen v \TeX u. Článek zpracovává téma, o kterém trojice \emph{sylabických univerzalistů} přednášela na TUGu 2023.
  \item \textbf{Co by každý \AllTeX ový nováček měl znát} od Barbary Beeton v českém překladu Honzy Šustka poskytuje užitečné rady pro začátečníky.
  \item \textbf{Sazba textu české lidové písně „Když jsem já
sloužil“ pomocí modulu l3seq jazyka expl3} od autora tohoto úvodníku ukazuje využití vysokoúrovňového programovacího jazyka expl3 na příkladu sazby textu české lidové písně.
\end{itemize}

Závěrem s radostí oznamuji, že konference TUG 2024 se uskuteční v Praze, což je historická událost, neboť od Euro\TeX u '92 se podobná akce v České republice nekonala. Konferenci organizuje společnost Overleaf za pomoci členů \CSTUG u. Aktuální informace o konferenci najdete na webových stránkách TUGu~\cite{tug2023tug} a v poštovním seznamu \texttt{cstug-members@cstug.cz}, jehož součástí jsou všichni členové \CSTUG u. Do poštovního seznamu směřujte také nabídky pomoci s organizací.

\printbibliography
 
\begin{summary}
The editorial presents an overview of the articles from this issue and announces TUG~2024, which will be held in Prague.
\end{summary}
\end{document}
